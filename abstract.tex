This is a general-purpose, simple exam generator for the \LaTeX\ documentclass 
exam~\cite{exam}.
It reads the question database from the source code of old exams or a file with 
questions.
The questions are expected to start with the \verb'\question' macro and end by 
the start of the next question, the \verb'\end{document}' macro or the end of 
the file.
In each question the generator will look for a label with the format 
\verb'\label{q:tag1:tag2:...:tagN}' or comments on the form
\verb'% tags: tag1:...:tagN'.
The set of tags is the union of all tags found.
The tags are used to filter which questions qualify for inclusion in the exam 
to be generated.
The tags can be used to represent topic, difficulty level, intended learning 
outcomes, etc.

The generation algorithm works as follows:
The set of desired tags, \(D\), is specified on the command line.
The set of tags of the generated exam, \(E\), is the union of the set of tags 
for each included question, \(Q_i\).
The union must be equal to the set of desired tags.
Thus \(E = \bigcup_i Q_i\) and \(E = D\) is the condition that must be 
fulfilled.
To achieve this, the algorithm randomly selects a question with tags \(Q_i\), 
if \(Q_i\subseteq E\) or \(Q_i\cap D\neq Q_i\) the question is discarded.
Otherwise the question adds one tag to \(E\) which brings \(E\) closer to 
equality of \(D\).
The algorithm ensures termination by either reaching equality or discarding all
questions in the database.

There is also an alternative interactive mode where each question is opened in 
the user's favourite editor.
When the editor exits the user is asked to accept or reject the question just 
edited.
