This is a general-purpose, simple exam generator for the \LaTeX\ documentclass 
exam~\cite{exam}.
It reads the question database from the source code of old exams, a file with 
questions or teaching material with exercises.
The questions are expected to be either questions from the exam document class, 
i.e.\ start with the
\verb'\question'
macro and end by the start of the next question, or be an exercise environment,
i.e.\ \verb'\begin{exercise}'
and
\verb'\end{exercise}'.

In each question the generator will look for tags.
Tags can be used to tag that a question covers a specific topic or intended 
learning outcome (or both).
The set of tags is the union of all tags found in the body of the question.
The tags are used to filter which questions qualify for inclusion in the exam 
to be generated.
Tags are given in questions as comments on the form
\verb'% tags: tag1:...:tagN',
and several such lines are allowed.

The generation algorithm then does the following:
\begin{enumerate}
\item Randomly select a question.
\item Apply a filter to the question; if the question passes, add it to the 
exam.
\item Check if the stop condition is reached; if not, start from the beginning.
\end{enumerate}
The default stop condition is as follows.
The set of desired tags, \(D\), is specified on the command line.
The set of tags of the generated exam, \(E\), is the union of the set of tags 
for each included question, \(Q_i\).
The union must be a superset of the set of desired tags.
We have \(E = \bigcup_i Q_i\) and, thus, \(E\supseteq D\) is the condition that 
must be fulfilled.
The algorithm ensures termination by either fulfilling the stop condition or 
discarding all questions in the database.

There is also an alternative interactive filter where each question is opened 
in the user's favourite editor.
When the editor exits the user is asked to accept or reject the question just 
edited.

The implementation is written as a highly modular library.
